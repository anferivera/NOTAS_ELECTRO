%\documentclass[letterpaper,12pt]{book}
%\usepackage[spanish]{babel}
%\usepackage{graphicx, color}
%\usepackage{amsmath}
%\usepackage{amscd}
%\usepackage{youngtab}
%\usepackage{titlesec}
%\usepackage[latin1]{inputenc}
%\usepackage{listings}
%\usepackage{multicol}
%\usepackage{enumerate}
%\usepackage{amssymb}
%\usepackage{amsthm}
%\usepackage{syntonly}
%\usepackage{fancyhdr}
%\begin{document}
\begin{thebibliography}{99}

\bibitem{alonso-z} Texto de Introducci�n a la F�sica

\bibitem{internet} http://www.unalmed.edu.co/fisica/

\bibitem{klepner} Introduction to Mechanics

\bibitem{observatorioP.A.} Jaime A. Mu�iz, Enrique Zas, {\textbf{El Observatorio de Rayos C�smicos Pierre Auger}}, \textit{Departamente de F�sica de Part�culas e Intituto Galego de Altas enerx�as, Universidade de Santiago de Compostela, 15782 Santiago, A Coru�a}, Febrero 6, 2008.

%\bibitem{remoto-tesis} Alberto Remoto, {\textbf{Weather efects on the cosmic ray energy spectrum with the surface detector ofthe Pierre Auger Observatory}}, FACOLT� DI SCIENZE MATEMATICHE, FISICHE E NATURALI, Universit� degli studi di Torino, 2009.\bibitem{observatorioP.A.} Jaime A. Mu�iz, Enrique Zas, {\textbf{El Observatorio de Rayos C�smicos Pierre Auger}}, \textit{Departamente de F�sica de Part�culas e Intituto Galego de Altas enerx�as, Universidade de Santiago de Compostela, 15782 Santiago, A Coru�a}, Febrero 6, 2008.
%
%\bibitem{bergstrom} L. Bergstr\"om, A. Goobar, {\textbf{COSMOLOGY AND PARTICLE ASTROPHYSICS}}, Stckholm, October 2003.
%
%\bibitem{gaisser} Gaisser Thomas K., {\textbf{Cosmic Rays and Particle Physics}}, Bartol research Institute, University of Delaware, Cambridge University Press 1990.
%
%\bibitem{correlation} The Pierre Auger Collaboration, et al, {\textbf{Correlation of Highest-Energy Cosmic Rays with Nearby Extragalactic Objets}}, Science \textbf{318}, 938 (2007).
%
%\bibitem{yoshida} Yoshida, Dai, J.Phys. G24 (1998) 905.
%
%\bibitem{geraldina} T. G. Golup, {\textbf{An�lisis de anisotrop�as en los datos del Observatorio Pierre Auger}}, Instituto Balseiro, Universidad Nacional de Cuyo, Comisi�n Nacional de Energ�a At�mica, Argentina, 2006.
%
%\bibitem{rielo} In�s Vali�o Rielo, {\textbf{DETECTION OF HORIZONTAL AND NEUTRINO INDUCED SHOWERS WITH THE PIERRE AUGER OBSERVATORY}},
% Universidad de Santiago de Compostela (Departamento de F�sica de Part�culas), diciembre, 2007.
% 
%\bibitem{antoni} T. Antoni et al., Astropart. Phys. {\textbf{14}} (2001) 245.
%
%\bibitem{grebe} Stefan Grebe, {\textbf{Properties of the surface detector data recorded by the Pierre Auger observatory}}, Universitat Siegen, Alemania, oktober, 2008.
% 
%\bibitem{isabell} Isabell Steinseifer, {\textbf{Energy calibration of the Pierre Auger Observatory using the constant intensity cut method}},
% Universitat Siegen, oktober, 2008.
%
%\bibitem{energy_dependence_of_the_CIC} Andr�s Rivera, Silvia Mollerach, Esteban Roulet, GAP 2009-123, {\textbf{On the energy dependence of the Constant Intensity Cut}}. 
%
%\bibitem{particle_distribution} E. Roulet, {\textbf{On particle distribution in extended air showers, weather effects and how to account for them}}, notas de estudio.
%
%\bibitem{spectrum} P. Auger Collaboration spectrum paper (2009).
%
%\bibitem{remoto} A. Remoto, S. Mollerach and E. Roulet, GAP 2009-054, {\textbf{``On the weather correction and energy calibration''}}
%
%\bibitem{on-CIC-and-weat-eff} E. Roulet, I. Allekotte, D. Harari and S. Mollerach, GAP 2009-024, {\textbf{``On the Constant Intensity Cut and weather effects''}}.
%
%\bibitem{weather} ``Atmospheric effects on extensive air showers observed with the Surface Detector of the Pierre Auger Observatory'', P. Auger Collaboration (2009), Astropart. Phys. in press.
%
%\bibitem{xmax} P. Auger Collaboration $X_{max}$ paper, in preparation (2009).
%
%\bibitem{geomagnetico} M. M\"unchmeyer, P. Billoir, A. Letessier, C. Bonifazi, GAP 2009-118, {\textbf{Monte Carlo studies of the influence of the geomagnetic field on air showers}}.
%
%\bibitem{newmethodsearch}  S. Mollerach, E. Roulet, {\textbf{A new method to search for a cosmic ray anisotropy}}. Journal of Cosmology and Astroparticle Physics.
%
%\bibitem{constraintslargeanisotropies} D. Harari, S. Mollerach and E. Roulet, GAP 2009-027, {\textbf{Constraints on large scale anisotropies from clean
%Rayleigh analyses.}}
% 
%\bibitem{zas} E. Zas, F. Halzen and T. Stanev, Phys. ReV. D \textbf{45} (1992) 362. 
%
%\bibitem{PDG} Particle Data Group, {\textbf{REVIEW OF PARTICLE
%PHYSICS}}, Experimental Methods and Colliders (passage of particles
%through matter).
%
%\bibitem{Ave} M. Ave, R. A. V�zquez, E. Zas, {\textbf{Modelling Horizontal Air Shower Induced by Cosmic Rays}}, arXiv:astro-hep-ph/0011490v1, 27 Nov 2000. \url{http://arxiv.org/abs/astro-ph/0011490}
%
%\bibitem{manual-panel} ISOFOTON, {\textbf{MANUAL DEL USUARIO DE
%M�DULOS FOTOVOLTAICOS}}.
%
%\bibitem{ANP} \url{http://pdg.lbl.gov/2007/AtomicNuclearProperties/index.html}
%
%\bibitem{FRD} \url{http://physics.nist.gov/PhysRefData/Star/Text/ESTAR.html}
%
%\end{thebibliography}

%\end{document}
