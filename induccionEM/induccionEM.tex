%\documentclass[letterpaper,12pt]{book}
%\usepackage[spanish]{babel}
%\usepackage{graphicx, color}
%\usepackage{amsmath}
%\usepackage{amscd}
%\usepackage{titlesec}
%\usepackage[latin1]{inputenc}
%\usepackage{listings}
%\usepackage{multicol}
%\usepackage{enumerate}
%\usepackage{amssymb}
%\usepackage{amsthm}
%\usepackage{syntonly}
%\usepackage{fancyhdr}
%\usepackage{verbatim}
%\usepackage{fancyvrb}
%\usepackage{hyperref}
%
%\begin{document}

%%%BEGIN+++++++++ induccionEM ++++++++++++++++++++

El principio de inducci�n fue formulado por Michael Faraday  (Inglaterra) e independientemente por Joseph Henry (Estados Unidos) en el a\~no de 1831.

\section{Flujo magn�tico}

\section{Principio de inducci�n}

%%%END+++++++++ induccionEM ++++++++++++++++++++++


%%%++++++++++++++++++++++++++++++++++++
%\end{document}


